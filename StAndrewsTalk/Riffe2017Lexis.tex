%% 
%%	This is file 'beamer_sample.tex'
%%	according to an MPIDR's PowerPoint template (?)
%%	
%%	by Eric Naujoks
%%
%%	Problems, bugs and comments to 
%%	naujoks@demogr.mpg.de
%%

%%%%%%%%%%%%%%%%%%%%%%%%%%%%%%%%%%
%%	Praelegomena								%%
%%%%%%%%%%%%%%%%%%%%%%%%%%%%%%%%%%
%%	- Make sure that you use utf8-encoding for all your .tex-files!!! (TeXnicCenter since version 2.0)
%%	- TeXnicCenter update: MPIDR intranet > Hard- & Sortfware > Software > Script and text editors > TeXnicCenter

\documentclass[20pt]{beamer}

\usepackage[ngerman,english]{babel}
\usepackage{tikz}
\usepackage[normalem]{ulem}
\geometry{paperwidth=10in, paperheight=7.5in}
\usepackage{animate}

\usepackage[utf8]{inputenc}

\usepackage[mpidr]{./mpidr/beamerthemeMPIDR}

%% Declaring title and author
\title{Extensions and applications of \\ the Lexis diagram}
\subtitle{Tim Riffe}		%%

%%	the institute's logo
\renewcommand{\mylogo}{\includegraphics[width=4.7in]{mpidr_logo_colour_en}}
\usepackage{color}
\definecolor{mygray}{rgb}{0.8,0.8,0.8}

\defbeamertemplate{description item}{align left}{\insertdescriptionitem\hfill}
%%	should be the very last package to be loaded
\usepackage{hyperref}

%%%%%%%%%%%%%%%%%%%%%%%%%%%%%%%%%%
%%	Beginning of the document		%%
%%%%%%%%%%%%%%%%%%%%%%%%%%%%%%%%%%
\begin{document}

%%	titlepage - fixed frame:
%%	========================

\begin{frame}
	\titlepage
\end{frame}
%-------------------

% TOC
%\begin{frame}{Plan}
%\tableofcontents
%\end{frame}

% -------------------
\section{Teaser}

\begin{frame}
\vspace{-15em}
\begin{center}
%\hspace*{-6cm}\includegraphics[scale=.7]{Figures/prism.jpg}
\vspace{4.25cm}
\hspace*{-6cm}\includegraphics[scale=1]{Figures/spacegrid.jpg}
\end{center}
\end{frame}
%

\begin{frame}
\vspace{-11em}
\begin{center}
%\hspace*{-6cm}\includegraphics[scale=.7]{Figures/prism.jpg}
\hspace*{-2cm}\includegraphics[scale=.9]{Figures/austro-hungarian-empire-railway-network-1912-2-1-smallmid-size.png}
\end{center}
\end{frame}
%

\section{Fast Intro}

\begin{frame}
\frametitle{The Lexis diagram}
\includegraphics[scale=1.5]{Figures/APCrt0.pdf}\\
\end{frame}

\begin{frame}
\frametitle{The Lexis diagram}
\includegraphics[scale=1.5]{Figures/APCrt1.pdf}\\
\end{frame}

\begin{frame}
\frametitle{The Lexis diagram}
\includegraphics[scale=1.5]{Figures/APCrt2.pdf}\\
\end{frame}

\begin{frame}
\frametitle{The Lexis diagram}
\includegraphics[scale=1.5]{Figures/APCrt3.pdf}\\
\end{frame}

\section{Standard applications}
%-----------------------------------------
%\begin{frame}
%\frametitle{Pattern detection \& storeytelling I}
%\includegraphics[scale=.7]{Figures/LungCancerDenmark.pdf}\\
%Lung cancer mortality, males, Denmark\\
%Data courtesy of Bendix Carstensen (2016)
%\end{frame}

\begin{frame}
\frametitle{Pattern detection \& storeytelling}
\includegraphics[scale=.85]{Figures/FertAPC.pdf}\\
Sweden fertility, 1891-2014 (HFD)
\end{frame}

\begin{frame}
\frametitle{Method design or explanation}
\includegraphics[scale=.65]{Figures/HMD_MPv5Fig4.png}\\
HMD Methods protocol (2007)
\end{frame}

%\begin{frame}
%\frametitle{Method design II}
%\includegraphics[scale=.7]{Figures/HMD_MPv5Fig8.png}\\
%HMD Methods protocol (2007)
%\end{frame}

%\begin{frame}
%\frametitle{Quality and consistency diagnostics}
%\includegraphics[scale=.8]{Figures/APC_males_CA_cancer.pdf}\\
%Cancer mortality, males, California.\\
%Data courtesy of HMD. 
%\end{frame}

\begin{frame}
\frametitle{Quality and consistency diagnostics}
\begin{center}
\includegraphics[scale=.8]{Figures/Schoeley2.png}
\end{center}
Sex differences in mortality\\The Human Mortality Explorer, Jonas Schoeley
(2015)
\end{frame}


%-----------------------------------------
\section{Non-standard applications}

\begin{frame}
\frametitle{Subpopulation composition and renewal}
\includegraphics[scale=.65]{Figures/SchmertmannJustices.png}
\end{frame}

\begin{frame}
\frametitle{Alternative axes: PhD Defense $=$ birth}
\includegraphics[scale=1.3]{Figures/LexisArticles.png}\\
Sula (2012)
\end{frame}

%\begin{frame}
%\frametitle{Cohort structure and composition}
%\includegraphics[scale=.55]{Figures/WPLexisPyramid.png}
%\end{frame}

%%-----------------------------------
%\subsection{aplicaciones no serias}
%\begin{frame}
%\frametitle{adivina que es esto I}
%\includegraphics[scale=.8]{Figures/PlayersAges.pdf}
%\end{frame}
%
%\begin{frame}
%\frametitle{adivina que es esto II}
%\includegraphics[scale=.88]{Figures/LexisPlayers1.pdf}\\
%datos scraped de Wikipedia
%\end{frame}
%
%\begin{frame}
%\frametitle{o primera entrada como el nacimiento}
%\includegraphics[scale=.88]{Figures/LexisPlayers2.pdf}\\
%pro tip: se puede hacer lo mismo con encuestas panel
%\end{frame}
%
%-----------------------------------
%\section{analogos}
%\begin{frame}
%\frametitle{otras relaciones como el EPC (APC)}
%\hspace{2cm}\includegraphics[scale=1.4]{Figures/TetraHedronEdgesOnly.pdf}\\
%Riffe, Schoeley, \& Villavicencio (2015)
%\end{frame}

%\begin{frame}
%\frametitle{el diagrama TAL}
%\hspace{2cm}\includegraphics[scale=.9]{Figures/TALrt.pdf}\\
%Riffe, Schoeley, \& Villavicencio (2015) \\ saltamos LCD y TPD \ldots
%\end{frame}

%\begin{frame}
%\frametitle{aplicaci\'{o}n del diagrama TAL I}
%\hspace{2cm}\includegraphics[scale=.9]{Figures/TAL_male_psych.pdf}\\
%Riffe, Chung, Spijker, \& MacInnes (2016)
%\end{frame}

%\begin{frame}
%\frametitle{aplicaci\'{o}n del diagrama TAL II}
%\hspace{2cm}\includegraphics[scale=.9]{Figures/TAL_male_back.pdf}\\
%Riffe, Chung, Spijker, \& MacInnes (2016)
%\end{frame}


%-----------------------------------
%\section{Transformations of Lexis}

%\begin{frame}
%\frametitle{Coordinate reprojection I}
%\includegraphics[scale=1]{Figures/ProspAgeLexis.pdf}\\
%$age \rightarrow e(age)$, Riffe (2016)
%\end{frame}

\begin{frame}
\frametitle{Coordinate reprojection}
\includegraphics[scale=.9]{Figures/FertAPC.pdf}\\
$~$\\
APC: Sweden fertility, 1891-2014 (HFD)
\end{frame}

\begin{frame}
\frametitle{Coordinate reprojection}
\includegraphics[scale=.9]{Figures/FertQuant.pdf}\\
$Age~\rightarrow~survival~quantile(Age)~\rightarrow~Fert(\alpha,t)$\\
Sweden fertility, 1891-2014 (HFD)
\end{frame}

%-----------------------------------
\section{Redux}
% strip Lexis back to two points and a duration.
\begin{frame}
\frametitle{The Lexis diagram}
\centering
\includegraphics[scale=1.5]{Figures/LexisStripped0.pdf}\\
\end{frame}

\begin{frame}
\frametitle{APC}
\centering
\includegraphics[scale=1.5]{Figures/LexisStripped1.pdf}\\
\end{frame}

\begin{frame}
\frametitle{PC vectors}
\centering
\includegraphics[scale=1.5]{Figures/LexisStripped2.pdf}\\
\end{frame}

\begin{frame}
\frametitle{PC as points on a timeline}
\centering
\includegraphics[scale=1.5]{Figures/PCline0.pdf}\\
\end{frame}

\begin{frame}
\frametitle{PC as points on a timeline}
\centering
\includegraphics[scale=1.5]{Figures/PCline1.pdf}\\
\end{frame}

\begin{frame}
\frametitle{Durations come from points}
\centering
\includegraphics[scale=1.5]{Figures/linep3.pdf}\\
\end{frame}

\begin{frame}
\frametitle{Durations come from points}
\centering
\includegraphics[scale=1.5]{Figures/linep4.pdf}\\
\end{frame}

\begin{frame}
\frametitle{Durations come from points}
\centering
\includegraphics[scale=1.5]{Figures/linep5.pdf}\\
\end{frame}

\begin{frame}
\frametitle{A graph representation}
\centering
\includegraphics[scale=1.5]{Figures/edgep2_1.pdf}\\
\end{frame}

\begin{frame}
\frametitle{A graph representation}
\centering
\includegraphics[scale=1.5]{Figures/edgep2_2.pdf}\\
\end{frame}

\begin{frame}
\frametitle{A graph representation}
\centering
\includegraphics[scale=1.5]{Figures/edgep3.pdf}\\
\end{frame}

\begin{frame}
\frametitle{A graph representation}
\centering
\includegraphics[scale=1.5]{Figures/edgep4.pdf}\\
\end{frame}

\begin{frame}
\frametitle{A graph representation}
\centering
\includegraphics[scale=1.1]{Figures/edgep5.pdf}\\
\end{frame}

\begin{frame}
\frametitle{A selection of properties of temporal identities}
\begin{enumerate}
  \item For $n$ events there are $m = \binom{n}{2}$ durations. (1,3,6,10,\ldots)
  \item The complete temporal graph, $G$, based on $n$ events has $n+1$ vertices
  and $n+m$ edges.
  \item There are $\binom{n+1}{3}$ triangle subidentities in $G$.
  (1,4,10,20,\ldots)
  \item $\binom{n}{2}$ of the subidentities have 2 events and 1 duration,
  like Lexis,  (1,3,6,10,\ldots)
  \item $\binom{n}{3}$ of the subidentities consist in 3 durations. (0,1,4,10)
  \item There are $(n+1)^{(n-1)}$ ways to fit a model \emph{covering} the $m+n$
  time measures implied by $n$ \emph{without} inducing a singularity. Everything else has
  APC-style redundancy.
\end{enumerate}
Note: not all durations scale like age, e.g. length of life. 
\end{frame}

\begin{frame}
\frametitle{Too much info?}
\centering
\huge{Time for an example}
\end{frame}

\begin{frame}
\frametitle{A demographic time framework}
\centering
\includegraphics[scale=1.5]{Figures/edgep3.pdf}\\
\end{frame}

\begin{frame}
\frametitle{A demographic time framework}
\centering
\includegraphics[scale=1.5]{Figures/APCTDLsq.pdf}\\
\end{frame}

\begin{frame}
\frametitle{A demographic time framework}
\centering
\vspace{-3em}
\includegraphics[scale=1.5]{Figures/TetraHedronEdgesOnly.pdf}\\
\end{frame}

\begin{frame}
\frametitle{A demographic time framework}
\centering
\vspace{-3em}
\includegraphics[scale=1.5]{Figures/TetraHedronEdgesOnlyTAL.pdf}\\
\end{frame}

\begin{frame}
\frametitle{A lifecourse diagram}
\centering
\includegraphics[scale=1]{Figures/TALrt.pdf}\\
\end{frame}

\begin{frame}
\frametitle{A lifecourse diagram: study area}
\centering
\includegraphics[scale=1]{Figures/TALHRS.pdf}\\
\end{frame}

\begin{frame}
\frametitle{A lifecourse diagram: smoothing fit}
\centering
\includegraphics[scale=1.5]{Figures/TetraHedronEdgesOnlyFit.pdf}\\
\end{frame}

\begin{frame}
\frametitle{Four major patterns}
\begin{centering}
\includegraphics[scale=.95]{Figures/Figure2a.pdf}\\
\end{centering}
``prevalence of pychological problems''\\
males, 1915-1919 cohort, USA (HRS)
\end{frame}

\begin{frame}
\frametitle{Four major patterns}
\begin{centering}
\includegraphics[scale=.95]{Figures/Figure2b.pdf}\\
\end{centering}
``prevalence of back problems''\\
females, 1915-1919 cohort, USA (HRS)
\end{frame}

\begin{frame}
\frametitle{Four major patterns}
\begin{centering}
\includegraphics[scale=.95]{Figures/Figure4a.pdf}\\
\end{centering}
``prevalence of smoking (ever)''\\
females, 1915-1919 cohort, USA (HRS)
\end{frame}

\begin{frame}
\frametitle{Four major patterns}
\begin{centering}
\includegraphics[scale=.95]{Figures/Figure4b.pdf}\\
\end{centering}
``prevalence of high blood pressure''\\
males, 1915-1919 cohort, USA (HRS)
\end{frame}





\end{document}
